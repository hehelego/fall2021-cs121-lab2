\documentclass{article}

\usepackage{xeCJK}
\usepackage{fancyhdr}
\usepackage{amsmath,amsthm,amssymb,amsfonts}
\usepackage{algorithm2e}
\usepackage{enumerate}
\usepackage{xifthen}
\usepackage{indentfirst}
\usepackage{xparse}
\usepackage{graphicx}
\usepackage{float}
\usepackage{xcolor}

\usepackage{hyperref}
\hypersetup{
    colorlinks=true,
    linkcolor=blue,
    filecolor=magenta,      
    urlcolor=cyan,
    pdftitle={Overleaf Example},
    pdfpagemode=FullScreen,
}
\urlstyle{same}

\usepackage[outputdir=tex-output]{minted}


\usepackage[
   backend=biber,
   style=alphabetic,
   sorting=ynt
]{biblatex}
\addbibresource{ref.bib}


\topmargin=-0.05in
\evensidemargin=0in
\oddsidemargin=0in
\textwidth=6.5in
\textheight=9.0in
\headsep=0.25in
\linespread{1.1}
\renewcommand\headrulewidth{0.2pt}
\renewcommand\footrulewidth{0.2pt}


%
% page style
%

\newcommand{\reportSection}[1]{
   \newpage
   \section{#1}
   \chead{section \emph{#1}}
}

\pagestyle{fancy}
\lhead{CS121 Lab2}
\rhead{\thepage}
\cfoot{\thepage}



%
% metadate for title page
%

\title{
   \textmd{\textbf{CS121@Fall2021 Lab 2\\ Cuckoo Hashing on GPU using CUDA}}\\
   \vspace{2in}
}
\author{
   {Cheng Peng (彭程)}\thanks{pengcheng2@shanghaitech.edu.cn}
   \and
   {2020533068}
}
\date{\today}

%
% alias
%

% argmax and argmin
\DeclareMathOperator*{\argmax}{arg\,max}
\DeclareMathOperator*{\argmin}{arg\,min}



%%%%%%%%%%%%%%%%%%%%%%
%                    %
%    the document    %
%                    %
%%%%%%%%%%%%%%%%%%%%%%


\begin{document}

%%%%% title page %%%%%
%%%%% TOC page %%%%%
\maketitle
\vfill
\tableofcontents
\vfill
\pagebreak

%%%%% content page %%%%%
\reportSection{Introduction}

Hash table is one of the most widely used data structure.
Since it was invented, extensive studies and practical uses have been made.
Separate chaining and open addressing.


Our overall goal is to make the algorithm as fast as possible.

\reportSection{The Cuckoo Hashing}

\reportSection{Algorithm Design}

\subsection{Choice of Hash Function}

To achieve good performance, we have to choose a set of hash function with care.
Hash functions of high quality are generally slow to compute, which may limit the throughput.
However, using fast hash functions may result in more collisions, which in turns lower our throughput.
Finding A hash function that achieves best balance between speed and quality is the key to high performance.

A research\cite{paper:hashfuncperfgpu} published on JCGT\footnote{Journal of Computer Graphics}
tested a set of cryptographic and non-cryptographic hash functions for their quality and speed in the context of GPU rendering.
They concluded that for 1D to 1D hash function, which is the scenario in lab2, the \emph{xxhash32} falls in the middle of the spectrum from the fastest algorithms to the algorithms that yield the best results.\par

A simple reference implementation\cite{blog:xxhimpl} in cpp was found.\par

The \emph{xxhash32} takes a two input a 4-byte seed and a byte stream, then produces a 32-bit integer as output.
Or formally speaking:

\[
   H:\mathrm{byte}^4\times \mathrm{byte}^\ast \to \mathrm{byte}^4
\]

Therefore, we can creat a set of hash functions by seeding \emph{xxhash32} with different seeds.

\[
   H_i:\mathrm{byte}^\ast \to \mathrm{byte}^4
   \qquad
   H_i(\mathrm{text}) = H(\mathrm{seed}_i,\mathrm{text})
\]

\reportSection{Performance Evaluation}

\reportSection{Futher Improvement}

%%%%% appendix page %%%%%
\appendix

\printbibliography[
	heading=bibintoc,
	title={bibliography}
]

\end{document}
